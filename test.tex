\documentclass{article}
\setlength{\parindent}{0pt}
\usepackage{graphicx}
\usepackage[T2A]{fontenc}
\usepackage[utf8]{inputenc}
\usepackage[russian]{babel}

\usepackage[left=2cm, top=2cm, right=2cm, bottom=2cm]{geometry}

\usepackage{breqn}\title{Матан Ботан}
\author{Кирилл Костышин}

\begin{document}
\maketitle

\pagebreak
\section{Основное уравнение}

Итак, нам дан такой пример: \\

\begin{dmath*}[spread=10pt]
f\left( x\right) = \left( \frac{ \cos{ \left( \arcsin{ \left( \cos{ x } \right) } \right) } }{ \cos{ x } + \tg{ x } } \right) ^{ \left( \ln{ \left( \cos{ x } \right) } \right) }
\end{dmath*}

Расчехляем дифференциатор и начинаем считать.

Посчитаем значение выражения:

\begin{gather*}
x = 1 \\
\end{gather*}

\begin{dmath*}
\left( \frac{ \cos{ \left( \arcsin{ \left( \cos{ x } \right) } \right) } }{ \cos{ x } + \tg{ x } } \right) ^{ \left( \ln{ \left( \cos{ x } \right) } \right) } = 1.75478
\end{dmath*}

\pagebreak
\section{Расчет производной}

Посчитаем производную 1-го порядка: \\

\begin{dmath*}[spread=10pt]
\frac{d}{dx} \left( x \right)= 1
\end{dmath*}

Так, я же нигде не обосрался? \textit{(© A. Скубачевский)} \\

\begin{dmath*}[spread=10pt]
\frac{d}{dx} \left( \cos{ x } \right)= \left( -1 \right) \cdot \sin{ x } \cdot 1
\end{dmath*}

Причёсываем это выражение \\

\begin{dmath*}[spread=10pt]
\frac{d}{dx} \left( \left( \ln{ \left( \cos{ x } \right) } \right) \right)= \frac{ 1 }{ \cos{ x } } \cdot \left( -1 \right) \cdot \sin{ x } \cdot 1
\end{dmath*}

Открываем БАЛШОЙ, БАЛШОЙ КВАДРАТНЫЙ СКОБКА \\

\begin{dmath*}[spread=10pt]
\frac{d}{dx} \left( x \right)= 1
\end{dmath*}

Это выражение из логарифмов ни уму ни сердцу ничего не говорит \\

\begin{dmath*}[spread=10pt]
\frac{d}{dx} \left( \cos{ x } \right)= \left( -1 \right) \cdot \sin{ x } \cdot 1
\end{dmath*}

Вы ещё скажите спасибо, что умножение коммутативно \\

\begin{dmath*}[spread=10pt]
\frac{d}{dx} \left( \arcsin{ \left( \cos{ x } \right) } \right)= \frac{ 1 }{ \left( 1 - \left( \cos{ x } \right) ^{ 2 } \right) ^{ 0.5 } } \cdot \left( -1 \right) \cdot \sin{ x } \cdot 1
\end{dmath*}

Любой уважающий себя синус трепыхается от -1 до 1. \\

\begin{dmath*}[spread=10pt]
\frac{d}{dx} \left( \cos{ \left( \arcsin{ \left( \cos{ x } \right) } \right) } \right)= \left( -1 \right) \cdot \sin{ \left( \arcsin{ \left( \cos{ x } \right) } \right) } \cdot \frac{ 1 }{ \left( 1 - \left( \cos{ x } \right) ^{ 2 } \right) ^{ 0.5 } } \cdot \left( -1 \right) \cdot \sin{ x } \cdot 1
\end{dmath*}

Любой уважающий себя синус трепыхается от -1 до 1. \\

\begin{dmath*}[spread=10pt]
\frac{d}{dx} \left( x \right)= 1
\end{dmath*}

Это выражение из логарифмов ни уму ни сердцу ничего не говорит \\

\begin{dmath*}[spread=10pt]
\frac{d}{dx} \left( \cos{ x } \right)= \left( -1 \right) \cdot \sin{ x } \cdot 1
\end{dmath*}

Ну всё, п***ц \textit{(© A. Скубачевский)} \\

\begin{dmath*}[spread=10pt]
\frac{d}{dx} \left( x \right)= 1
\end{dmath*}

Так, я же нигде не обосрался? \textit{(© A. Скубачевский)} \\

\begin{dmath*}[spread=10pt]
\frac{d}{dx} \left( \tg{ x } \right)= \frac{ 1 }{ \left( \cos{ x } \right) ^{ 2 } } \cdot 1
\end{dmath*}

Ну всё, п***ц \textit{(© A. Скубачевский)} \\

\begin{dmath*}[spread=10pt]
\frac{d}{dx} \left( \cos{ x } + \tg{ x } \right)= \left( -1 \right) \cdot \sin{ x } \cdot 1 + \frac{ 1 }{ \left( \cos{ x } \right) ^{ 2 } } \cdot 1
\end{dmath*}

Переменную интегрирования можно обозначить любой буквой: $x$, $y$, й...  \\

\begin{dmath*}[spread=10pt]
\frac{d}{dx} \left( \left( \frac{ \cos{ \left( \arcsin{ \left( \cos{ x } \right) } \right) } }{ \cos{ x } + \tg{ x } } \right) \right)= \frac{ \left( -1 \right) \cdot \sin{ \left( \arcsin{ \left( \cos{ x } \right) } \right) } \cdot \frac{ 1 }{ \left( 1 - \left( \cos{ x } \right) ^{ 2 } \right) ^{ 0.5 } } \cdot \left( -1 \right) \cdot \sin{ x } \cdot 1 \cdot \left( \cos{ x } + \tg{ x } \right) - \cos{ \left( \arcsin{ \left( \cos{ x } \right) } \right) } \cdot \left( \left( -1 \right) \cdot \sin{ x } \cdot 1 + \frac{ 1 }{ \left( \cos{ x } \right) ^{ 2 } } \cdot 1 \right) }{ \left( \cos{ x } + \tg{ x } \right) ^{ 2 } }
\end{dmath*}

Вы ещё скажите спасибо, что умножение коммутативно \\

По итогу получаем:

\begin{dmath*}[spread=10pt]
\frac{d}{dx} \left( \left( \frac{ \cos{ \left( \arcsin{ \left( \cos{ x } \right) } \right) } }{ \cos{ x } + \tg{ x } } \right) ^{ \left( \ln{ \left( \cos{ x } \right) } \right) } \right)= \left( \frac{ \cos{ \left( \arcsin{ \left( \cos{ x } \right) } \right) } }{ \cos{ x } + \tg{ x } } \right) ^{ \left( \ln{ \left( \cos{ x } \right) } \right) } \cdot \left( \frac{ 1 }{ \cos{ x } } \cdot \left( -1 \right) \cdot \sin{ x } \cdot 1 \cdot \ln{ \left( \frac{ \cos{ \left( \arcsin{ \left( \cos{ x } \right) } \right) } }{ \cos{ x } + \tg{ x } } \right) } + \ln{ \left( \cos{ x } \right) } \cdot \frac{ \frac{ \left( -1 \right) \cdot \sin{ \left( \arcsin{ \left( \cos{ x } \right) } \right) } \cdot \frac{ 1 }{ \left( 1 - \left( \cos{ x } \right) ^{ 2 } \right) ^{ 0.5 } } \cdot \left( -1 \right) \cdot \sin{ x } \cdot 1 \cdot \left( \cos{ x } + \tg{ x } \right) - \cos{ \left( \arcsin{ \left( \cos{ x } \right) } \right) } \cdot \left( \left( -1 \right) \cdot \sin{ x } \cdot 1 + \frac{ 1 }{ \left( \cos{ x } \right) ^{ 2 } } \cdot 1 \right) }{ \left( \cos{ x } + \tg{ x } \right) ^{ 2 } } }{ \frac{ \cos{ \left( \arcsin{ \left( \cos{ x } \right) } \right) } }{ \cos{ x } + \tg{ x } } } \right) = \left( \frac{ \cos{ \left( \arcsin{ \left( \cos{ x } \right) } \right) } }{ \cos{ x } + \tg{ x } } \right) ^{ \left( \ln{ \left( \cos{ x } \right) } \right) } \cdot \left( \frac{ 1 }{ \cos{ x } } \cdot \left( -1 \right) \cdot \sin{ x } \cdot \ln{ \left( \frac{ \cos{ \left( \arcsin{ \left( \cos{ x } \right) } \right) } }{ \cos{ x } + \tg{ x } } \right) } + \ln{ \left( \cos{ x } \right) } \cdot \frac{ \frac{ \left( -1 \right) \cdot \sin{ \left( \arcsin{ \left( \cos{ x } \right) } \right) } \cdot \frac{ 1 }{ \left( 1 - \left( \cos{ x } \right) ^{ 2 } \right) ^{ 0.5 } } \cdot \left( -1 \right) \cdot \sin{ x } \cdot \left( \cos{ x } + \tg{ x } \right) - \cos{ \left( \arcsin{ \left( \cos{ x } \right) } \right) } \cdot \left( \left( -1 \right) \cdot \sin{ x } + \frac{ 1 }{ \left( \cos{ x } \right) ^{ 2 } } \right) }{ \left( \cos{ x } + \tg{ x } \right) ^{ 2 } } }{ \frac{ \cos{ \left( \arcsin{ \left( \cos{ x } \right) } \right) } }{ \cos{ x } + \tg{ x } } } \right)
\end{dmath*}

Всё, что недосократилось, сократите сами, РУЧКАМИ

\end{document}

